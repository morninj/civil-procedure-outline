\section{The Governing Law in the Federal Courts}

\subsection{\emph{Swift v. Tyson}}

\begin{enumerate}
    \item Swift owned a bill of exchange that Tyson originally made to two other men, who endorsed it over to Swift. Tyson refused to pay because of a breach of the original contract for which the bill of exchange was originally issued. Swift sued Tyson in federal court for payment. Under ``local`` law (New York state law), which the state court followed, Tyson's defense was valid. Under ``general'' law, which federal courts followed, the defense was not valid against Swift.
    \item § 34 of the Judiciary Act of 1789 required the application of ``the laws of the several states.'' (Roughly equivalent to 28 U.S.C. §1652.)
    \item In a unanimous opinion, Justice Story held that the federal court ``should follow the general law rather than a state's local law in cases where the state law deviated from the general law.''\footnote{Casebook p. 491.}
    \item After the Civil War, the economic interests of the states began to diverge (north: finance, manufacturing; south: agriculture, labor). The Supreme Court expanded general law to include torts, so ndustrial accidents were increasingly litigated in federal courts. Federal courts also grew increasingly sympathetic to creditors and employers. Those who favored the results of state courts became enemies of \emph{Swift}.
\end{enumerate}

\subsection{\emph{Erie R.R. v. Tompkins}}

\begin{enumerate}
    \item A passing train injured the plaintiff was while he was walking on a footpath in Pennsylvania. He brought suit in federal court in Southern New York. Under Pennsylvania state law, the plaintiff would have been considered a trespasser and therefore not entitled to recover damages. Under general law, the railroad might have been held negligent.
    \item The trial court and appellate court, relying on general law, found for the plaintiff. The issue before the Supreme Court was whether the federal court was free to disregard the rules of Pennsylvania common law.
    \item The court found that the \emph{Swift} court had misinterpreted the intentions of the authors of the Judiciary Act of 1789: ``...the construction given to it by the court was erroneous; and that the purpose of the section was merely to make certain that, in all matters except those in which some federal law is controlling, the federal courts exercising jurisdiction in diversity of citizenship cases would apply as their rules of decision the law of the state, unwritten as well as written.''
    \item \emph{Swift} had caused a range of devious legal maneuvering---e.g., companies reincorporating in other states in order to establish diversity jurisdiction to have their cases tried in federal court (\emph{Black \& White Taxicab}). ``\emph{Swift v. Tyson} introduced grave discrimination by noncitizens against citizens.''\footnote{Casebook p. 496.}
    \item \emph{Swift} caused a divergence between state and federal law, and it was constantly difficult to demarcate areas where federal law and state law applied.
    \item ``Except in matters governed by the Federal Constitution or by acts of Congress, the law to be applied in any case is the law of the state.... There is no federal general common law.''\footnote{Casebook p. 497.} Judges often rely on ``general law'' as a way of ignoring state laws that conflict with their views.
    \item Common law is the common law of the \emph{state}, not of ``general law'' or of England or of anywhere else.
    \item Judgment is reversed and remanded to be decided on the basis of state law.
    \item Justice Reed, concurring: it is enough to broaden the \emph{Swift} framework to include state common law, rather than declare it unconstitutional.
\end{enumerate}

\subsection{\emph{Guaranty Trust v. York}}

\begin{enumerate}
    \item Background:
    \begin{enumerate}
        \item \textbf{Substance vs. procedure}: did \emph{Erie} and the Rules of Decision Act apply to procedural matters in addition to substantive law?
        \item Before the FRCP in 1938, courts were divided into courts of law (in which cases were triable by jury) and courts of equity (where there was no right to a jury).
        \item The \textbf{Conformity Act of 1872} required that in cases \emph{at law}, federal courts must conform to the procedural rules of the states in which they are located. Thus, there were procedural differences between federal courts in different states, but few differences between federal and state courts in the same state.
        \item But, in cases \emph{in equity}, federal and state courts followed different rules. Federal courts developed their own system of procedural rules for suits brought in equity.\footnote{Casebook p. 504.}
        \item The \textbf{Rules Enabling Act of 1934} authorized the Supreme Court to develop a national system of procedural rules for federal civil cases. So while \emph{Erie} required federal courts to follow states in substantive law, federal courts developed independent rules (codified in the FRCP) for procedural law. 
    \end{enumerate}
\end{enumerate}

\subsection{\emph{Byrd v. Blue Ridge Rural Elec. Coop.}}

\begin{enumerate}
    \item todo
\end{enumerate}
