\section{The Governing Law in the Federal Courts}

\subsection{\emph{Swift v. Tyson}}

\begin{enumerate}
    \item Swift owned a bill of exchange that Tyson originally made to two other men, who endorsed it over to Swift. Tyson refused to pay because of a breach of the original contract for which the bill of exchange was originally issued. Swift sued Tyson in federal court for payment. Under ``local`` law (New York state law), which the state court followed, Tyson's defense was valid. Under ``general'' law, which federal courts followed, the defense was not valid against Swift.
    \item § 34 of the Judiciary Act of 1789 required the application of ``the laws of the several states.'' (Roughly equivalent to 28 U.S.C. §1652.)
    \item In a unanimous opinion, Justice Story held that the federal court ``should follow the general law rather than a state's local law in cases where the state law deviated from the general law.''\footnote{Casebook p. 491.}
    \item After the Civil War, the economic interests of the states began to diverge (north: finance, manufacturing; south: agriculture, labor). The Supreme Court expanded general law to include torts, so ndustrial accidents were increasingly litigated in federal courts. Federal courts also grew increasingly sympathetic to creditors and employers. Those who favored the results of state courts became enemies of \emph{Swift}.
\end{enumerate}

\subsection{\emph{Erie R.R. v. Tompkins}}

\begin{enumerate}
    \item A passing train injured the plaintiff, Tompkins, was while he was walking on a footpath along a railroad track in Pennsylvania. He brought suit in federal court in Southern New York. Under Pennsylvania state law, the plaintiff would have been considered a trespasser and therefore not entitled to recover damages. Under general law, the railroad might have been held negligent.
    \item The legal circumstances were unusual. At the time, ``general law'' usually benefited corporations, while ``local law'' usually favored individuals. In this case, however, the plaintiff argued for the application of general law.
    \item The trial court and appellate court, following \emph{Swift}, found that since no state statute governed the issue at hand, general law should control. They found for the plaintiff. The issue before the Supreme Court was whether the district court was free to disregard Pennsylvania common law.
    \item The Court overturned the \emph{Swift} rule, After \emph{Erie}, there was only federal law and state law. Wherever the federal law did not explicitly apply, state law controlled.
    \item Justice Brandeis made three arguments:
    \begin{enumerate}
        \item First: the \emph{Swift} court had misinterpreted the intentions of the authors of the Judiciary Act of 1789: ``...the construction given to it by the court was erroneous; and that the purpose of the section was merely to make certain that, in all matters except those in which some federal law is controlling, the federal courts exercising jurisdiction in diversity of citizenship cases would apply as their rules of decision the law of the state, unwritten as well as written.'' (The notes point out that the historian on whom the opinion relies, Charles Warren, might have mistakenly interpreted the Rules of Decision Act as requiring federal courts to follow state common law even in areas where federal common law applied.)
        \item Second: \emph{Swift} caused significant ``injustice and confusion''\footnote{Casebook p. 501}---e.g., companies reincorporating in other states in order to establish diversity jurisdiction to have their cases tried in federal court (\emph{Black \& White Taxicab}). ``\emph{Swift v. Tyson} introduced grave discrimination by noncitizens against citizens.''\footnote{Casebook p. 496.}
        \item Third: The federal government did not have the power to legislate rules of tort or contract law. (This quickly became untrue as the Court expanded the federal government's power to regulate these areas under the Commerce Clause). Federal courts also do not have the power to create rules in these areas.
    \end{enumerate}
    \item ``Except in matters governed by the Federal Constitution or by acts of Congress, the law to be applied in any case is the law of the state.... There is no federal general common law.''\footnote{Casebook p. 497.} Judges often rely on ``general law'' as a way of ignoring state laws that conflict with their views.
    \item Judgment was reversed and remanded to be decided on the basis of Pennsylvania state law.
    \item Justice Reed, concurring:
    \begin{enumerate}
        \item It is enough to broaden the \emph{Swift} framework to hold that ``the laws'' in the Rules of Decison Act include state common law, rather than declare the entire \emph{Swift framework} to be unconstitutional.
        \item It's ``questionable'' to say that Congress has no power to declare which substantive laws contol in federal courts---moreso because ``[t]he line between procedural and substantive law is hazy.''\footnote{Casebook p. 499.}
    \end{enumerate}
\end{enumerate}

\subsection{``Outcome Determinative'' Test: \emph{Guaranty Trust v. York}}

\begin{enumerate}
    \item Background:
    \begin{enumerate}
        \item \textbf{Substance vs. procedure}: did \emph{Erie} and the Rules of Decision Act apply to both procedural and substantive law?
        \item Before the FRCP were enacted in 1938, courts were divided into courts of law (in which cases were triable by jury) and courts of equity (where there was no right to a jury).
        \item The \textbf{Conformity Act of 1872} required that in cases \emph{at law}, federal courts must conform to the procedural rules of the states in which they are located. Thus, there were procedural differences between federal courts in different states, but few differences between federal and state courts in the same state.
        \item But, in cases \emph{in equity}, federal and state courts followed different rules. Federal courts developed their own system of procedural rules for suits brought in equity.\footnote{Casebook p. 504.}
        \item The \textbf{Rules Enabling Act of 1934} authorized the Supreme Court (with congressional approval) to develop a national system of procedural rules for federal civil cases. So while \emph{Erie} required federal courts to follow state rules in substantive law, federal courts developed independent procedural rules (codified in the FRCP in 1938).
    \end{enumerate}
    \item 1942: The plaintiff, York, brought suit in equity in NY federal court for fraud that occurred in 1931. The defendant argued that the NY statute of limitations applied both to cases at law and in equity. The plaintiff argued that the federal rule of laches, which typically applied in equity cases, should apply in this case.
    \item The trial court applied the NY statute of limitations. The appellate court reversed, holding that the laches doctrine should have applied, and granted summary judgment to the defendants.
    \item Justice Frankfurter:
    \begin{enumerate}
        \item There is not a clear distinction between ``substantive'' and ``procedural'' rights.
        \item This case dealt with a state-created right. When a federal court adjudicates a state-created right solely on the basis of jurisdiction, it becomes ``in effect, only another court of the State.''
        \item  Question: does the statute of limitations affect ``merely the manner and the means'' of the right to recover, or is it ``a matter of substance'' that affects the result of the litigation?
        \item \emph{Erie} did not intend to ``formulate scientific legal terminology'' around the terms ``substantive'' and ``procedural.'' It intended to ensure that outcomes of diversity cases in federal court would be similar to outcomes in state courts.
        \item The \emph{Erie} doctrine does not distinguish between cases at law and cases in equity.
        \item ``The source of substantive rights enforced by a federal court under diversity jurisdiction, it cannot be said too often, is the law of the States. Whenever that law is authoritatively declared by a State, whether its voice be the legislature or its highest court, such law ought to govern in litigation founded on that law, whether the forum of application is a State or a federal court and whether the remedies be sought at law or may be had in equity.''\footnote{Casebook p. 507.}
        \item The Court reversed the appellate. Remanded for hearings under the NY statute of limitations.
    \item The \textbf{``outcome determinative'' test}: would it significantly affect the outcome of the litigation for a federal court to disregard state law? If so, \emph{Erie} holds that the court should follow state law.
    \end{enumerate}
    \item Justice Rutledge, dissenting:
    \begin{enumerate}
        \item The distinction between ``substantive'' and ``procedural'' law is arbitrary but important.
        \item Forum states are free to apply their own statutes of limitations, which may be different from those of the state that originally created the substantive right.
    \end{enumerate}
\end{enumerate}

\subsection{Challenging the Rules Enabling Act: \emph{Sibbach v. Wilson \& Co.}}

\begin{enumerate}
    \item A federal court required a plaintiff to submit to a medical examination. The examination would not have been mandatory under Illinois state law. She refused the exam on the grounds that the Rules Enabling Act forbade rules that abridge litigants' substantive rights. The court held that the rule requiring medical examinations was procedural, not substantive.
    \item Justice Frankfurter (and three others) dissented, arguing that the examination rule constituted ``invasion of the person'' and was therefore significantly different from other procedural rules.
    % TODO: add notes pp. 509-511
\end{enumerate}

\subsection{\emph{Byrd v. Blue Ridge Rural Elec. Coop.}}

\begin{enumerate}
    \item The plaintiff, Byrd, sued the defendant for negligence for an injury he sustained while connecting power lines.
    \item The defendant argued that the plaintiff was a ``statutory employee'' under the South Carolina Workmen's Compensation Act when the injury occurred, which would mean that the plaintiff was barred from suing and was obliged to accept statutory compensation benefits.
    \item The questions on appeal are (1) whether the appellate court erred in directing judgment for Blue Ridge without giving Byrd an opportunity to introduce further evidence, and (2) whether Byrd is entitled to a jury to determine factual issues.
    \item Justice Brennan:
    \begin{enumerate}
        \item Blue Ridge argued that a judge, not a jury, should decide the question of immunity. In \emph{Adams v. Davidson-Paxon Co.}, the South Carolina Supreme court held that a judge, not a jury, should determine the question of whether the plaintiff was a statutory employee (and therefore whether the employer is immune from paying damages). The defendant's argument contends that the federal court should follow this state precedent.
        \item First: \emph{Erie} held that in diversity cases, federal courts must respect state-created rights by state courts. Here, the Supreme Court found that the decision in \emph{Adams} to send the immunity question to a judge was a ``practical consideration''---``merely a form and mode''---``and not a rule intended to be bound up with the definition of the rights and obligations of the parties.''
        \item Second: Mere ``form and mode'' rules can be important if they bear substantially on the outcome of the litigation. It may be in this case that the question of whether immunity should be decided by a judge or jury would bear substantially on the outcome. Here, however, there is a general federal policy of sending factual questions to the jury. Even though it may affect the outcome, the federal rule should outweigh the state rule because a federal procedural rule is ``not in any sense a local matter.''
        \item Third, it is not at all clear that sending the immunity question to a jury would result in a different outcome. % TODO: expand
        \item The federal rule applies. Reversed and remanded.
    \end{enumerate}
\end{enumerate}

\subsection{\emph{Hanna v. Plumer}}

\begin{enumerate}
    \item todo

% two tests: hanna holding and hanna dictum
% see bradt class notes

\end{enumerate}

\subsection{\emph{Walker v. Armco Steel Corp.}}

\begin{enumerate}
    \item todo
\end{enumerate}

\subsection{\emph{Gasperini v. Center for Humanities}}

\begin{enumerate}
    \item todo
\end{enumerate}

\subsection{\emph{Clearfield Trust Co v. United States}}

\begin{enumerate}
    \item todo
\end{enumerate}
