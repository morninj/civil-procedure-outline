\section{Preclusion}

\begin{enumerate}
    \item \textbf{Res judicata} (claim preclusion): a claim (and related 
    claims) cannot be relitigated after final judgment.\footnote{Casebook p. 
    1224.}
    \item \textbf{Collateral estoppel} (issue preclusion): an issue of fact or 
    law cannot be relitigated after final judgment.
    \item When filing a claim, failing to file a related claim can prevent 
    the claimant from litigating that claim in the future.
\end{enumerate}

\subsection{\emph{Federated Department Stores, Inc. v. Moitie}}

Once a claim reaches final judgment, parties cannot raise other claims arising 
from the same transaction or occurrence. It is usually worth keeping a case 
alive on appeal.

\begin{enumerate}
    \item The government brought an antitrust suit against Federated 
    Department Stores and others. Seven plaintiffs filed civil actions, 
    including Moitie in state court (\emph{Moitie I}) and Brown in the 
    Northern District of California (\emph{Brown I}). The civil claims followed the govenment's 
    claims almost verbatim, although Moitie referred only to state law.
    \item \emph{Moitie I} was removed to district court. All civil claims were 
    directed to the same federal judge.
    \item The district court rejected all of the civil claims for failure to 
    allege an ``injury'' to their ``business or property'' under \S\ 4 of the 
    Clayton Act.\footnote{15 U.S.C. \S\ 15.}
    \item Five of the plaintiffs appelaed in the Ninth Circuit. The lawyer 
    representing Moitie and Brown, however, chose to refile in state court 
    (\emph{Moitie II} and \emph{Brown II}).
    \item \emph{Moitie II} and \emph{Brown II} were removed to federal court. 
    The court found that they were ``in many respects identical'' to the 
    earlier claims. It dismissed them under res judicata.\footnote{Casebook p. 
    1225.}
    \item While \emph{Moitie II} and \emph{Brown II} were pending appeal, the 
    Supreme Court decided \emph{Reiter v. Sonatone Corp.}, holding that 
    retailers \emph{could} allege an ``injury'' to their ``business or 
    property'' under \S\ 4 of the Clayton Act, and accordingly the Ninth 
    Circuit reversed the five cases pending appeal. The Ninth Circuit also 
    reversed the dismissals of \emph{Moitie II} and \emph{Brown II} on the 
    same grounds, even though it violated a strict interpretation of res 
    judicata, becuase ``the doctrine of res judicata must give way to `public 
    policy' and `simple justice.'''\footnote{Casebook p. 1226.}
    \item Justice Rehnquist:
    \begin{enumerate}
        \item ``...such an unwarranted departure from res judicata is 
        unwarranted. Indeed, the decision below is all but foreclosed by our 
        prior case law.''\footnote{Casebook p. 1227.}
        \item ``The doctrine of res judicata serves vital public interests 
        beyond any individual judge's determination of the equities in a 
        particular case.''\footnote{Casebook p. 1229.}
        \item Reversed.
    \end{enumerate}
    \item Justice Blackmun, concurring:
    \begin{enumerate}
        \item There may be cases where policy concerns override res judicata.
        \item \emph{Brown II} should not even been allowed in federal court 
        because \emph{Brown I} was res judicata.
    \end{enumerate}
    \item Justice Brennan, dissenting:
    \begin{enumerate}
        \item Agree with Blackmun that \emph{Brown I} was res judicata.
    \end{enumerate}
\end{enumerate}
