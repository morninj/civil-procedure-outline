\section{Preclusion}

\begin{enumerate}
    \item \textbf{Res judicata} (claim preclusion): a claim (and related 
    claims) cannot be relitigated after final judgment.\footnote{Casebook p.  
    1224.}
    \item \textbf{Collateral estoppel} (issue preclusion): an issue of fact or 
    law cannot be relitigated after final judgment.
    \item When filing a claim, failing to file a related claim can prevent the 
    claimant from litigating that claim in the future.
    \item Some states define ``claim'' more narrowly--e.g., California relies 
    on the ``primary right,'' allowing two suits based on the same facts to be 
    brought separately if they are based on the same primary 
    right.\footnote{Casebook p. 1238.}
    \item Restatement (Second) of Judgments, \S\ 24(2): ``What factual 
    grouping constitutes a `transaction,' and what groupings constitute a 
    `series,' are to be determined pragmatically, giving weight to such 
    considerations as whether the facts are related in time, space, origin, or 
    motivation, whether they form a convenient trial unit, and whether their 
    treatment as a unit conforms to the parties' expectations or business 
    understanding or usage.''\footnote{Casebook p. 1239.}
    \item Failure to assert a compulsory counterclaim generally bars that 
    claim from future actions.\footnote{Casebook p. 1240.}
    \item Intervenors must pursue its claims or be barred by claim preclusion 
    in future actions.\footnote{Casebook p. 1240.}
    \item Res judicata generally applies to causes of action but not to 
    parties.\footnote{Casebook p. 1241.}
    \item A claim is precluded if a court has entered final judgment on the 
    merits. The definitive final judgment is a trial verdict affirmed on 
    appeal. \textbf{Non-final judgments} include dismissals for problems with 
    subject matter jurisdiction, personal jurisdiction, or venue. Most other 
    types of judgment (failure to state a claim, summary judgment, dismissal 
    for failure to follow court orders) generally count as final. The question 
    is whether the plaintiff had an opportunity to be heard on the merits.
\end{enumerate}

\subsection{Consequences of Final Judgment: \emph{Federated Department Stores, 
Inc. v. Moitie}}

Once a claim reaches final judgment, parties cannot raise other claims arising 
from the same transaction or occurrence. Res judicata overrides competing 
policy concerns. It is usually worth keeping a case alive on appeal.

\begin{enumerate}
    \item The government brought an antitrust suit against Federated 
    Department Stores and others. Seven plaintiffs filed civil actions, 
    including Moitie in state court (\emph{Moitie I}) and Brown in the 
    Northern District of California (\emph{Brown I}). The civil claims 
    followed the govenment's claims almost verbatim, although Moitie referred 
    only to state law.
    \item \emph{Moitie I} was removed to district court. All civil claims were 
    directed to the same federal judge.
    \item The district court rejected all of the civil claims for failure to 
    allege an ``injury'' to their ``business or property'' under \S\ 4 of the 
    Clayton Act.\footnote{15 U.S.C. \S\ 15.}
    \item Five of the plaintiffs appelaed in the Ninth Circuit. The lawyer 
    representing Moitie and Brown, however, chose to refile in state court 
    (\emph{Moitie II} and \emph{Brown II}).
    \item \emph{Moitie II} and \emph{Brown II} were removed to federal court.  
    The court found that they were ``in many respects identical'' to the 
    earlier claims. It dismissed them under res judicata.\footnote{Casebook p.  
    1225.}
    \item While \emph{Moitie II} and \emph{Brown II} were pending appeal, the 
    Supreme Court decided \emph{Reiter v. Sonatone Corp.}, holding that 
    retailers \emph{could} allege an ``injury'' to their ``business or 
    property'' under \S\ 4 of the Clayton Act, and accordingly the Ninth 
    Circuit reversed the five cases pending appeal. The Ninth Circuit also 
    reversed the dismissals of \emph{Moitie II} and \emph{Brown II} on the 
    same grounds, even though it violated a strict interpretation of res 
    judicata, becuase ``the doctrine of res judicata must give way to `public 
    policy' and `simple justice.'''\footnote{Casebook p. 1226.}
    \item Justice Rehnquist:
    \begin{enumerate}
        \item ``...such an unwarranted departure from res judicata is 
        unwarranted. Indeed, the decision below is all but foreclosed by our 
        prior case law.''\footnote{Casebook p. 1227.}
        \item ``The doctrine of res judicata serves vital public interests 
        beyond any individual judge's determination of the equities in a 
        particular case.''\footnote{Casebook p. 1229.}
        \item Reversed.
    \end{enumerate}
    \item Justice Blackmun, concurring:
    \begin{enumerate}
        \item There may be cases where policy concerns override res judicata.
        \item \emph{Brown II} should not even been allowed in federal court 
        because \emph{Brown I} was res judicata.
    \end{enumerate}
    \item Justice Brennan, dissenting:
    \begin{enumerate}
        \item Agree with Blackmun that \emph{Brown I} was res judicata.
    \end{enumerate}
\end{enumerate}

\subsection{Claim Preclusion: \emph{Davis v. DART}}

A plaintiff must bring all causes of action related to the same claim. Any 
related actions he fails to bring are barred from future suits. Preclusion is 
harsh.

\begin{enumerate}
    \item 2001: Davis and Johnson alleged race discrimination and retaliation 
    under Title VII and violations of the First and Fourteenth Amendments 
    under 42 U.S.C. \S\ 1983 against Dallas Area Rapid Transit and its Chief 
    of Police. They originally brought the claim in state court and it was 
    removed to Texas district court (\emph{Davis I}). The district court 
    dismissed the claims with prejudice.
    \item 2002: Davis and Johnson brought another action in district court 
    alleging similar (but not identical) claims. The court granted summary 
    judgment for the defendants on the grounds that (1) they failed to raise 
    an issue of fact about whether their nonselection for promotion was 
    racially motivated and (2) res judicata from \emph{Davis I} precluded the 
    remaining claims.
    \item The Fifth Circuit identified four factors for barring claims under 
    res judicata:
    \begin{enumerate}
        \item Identical parties.
        \item Prior judgment from a court of competent jurisdiction.
        \item Prior judmgent that was final and on the merits.
        \item Same cause of action in both suits.
    \end{enumerate}
    \item Only the fourth factor was disputed here. The standard of review for 
    the Fifth Circuit was (1) whether the barred claims were part of the same 
    cause of action (``same nucleus of operative facts'') and (2) whether 
    Davis and Johnson could have advances the barred claims in \emph{Davis 
    I}.\footnote{Casebook p. 1233.}
    \item The court here found that the claims in both cases ``originate from 
    the same continuing course of allegedly discriminatory conduct'' and that 
    the claims could have been brought together (despite the plaintiff's 
    argument that their pending EEOC claim prevented bringing the full action 
    in court).
\end{enumerate}

\subsection{Claim Splitting: \emph{Staats v. County of Sawyer}}

If the initial forum lacks jurisdiction to adjudicate the entirety of the 
plaintiff's claims, the remaining claims are not barred from future 
litigation.

\begin{enumerate}
    \item The county eliminated Staats' job. He believed the county based its 
    decision on disability discrimination. He filed a state law claim with the 
    Wisconsin Equal Rights Division, which authorized a hearing before an 
    administrative judge. That judge found for Staats. On appeal, the Labor 
    and Industry Review Commission rejected his claim and the state court 
    affirmed.
    \item Meanwhile, he filed charges with federal Equal Employment 
    Opportunity Commission, which issued him a right-to-sue letter. The 
    district court rejected his federal claims on the ground of claim 
    preclusion.
    \item The Seventh Circuit here held that the initial state forum 
    did not have jurisdiction to hear the entirety of Staats' claim. Rather, 
    the circumstances required \textbf{claim splitting} between that forum and 
    another. ``...Staats had no way to consolidate his WFEA, ADA, and 
    Rehabilitation Act claims in any single forum. He was forced to split his 
    claims and litigate them in separate fora.''\footnote{Casebook p. 1246.} 
    Staats was not precluded from bringing his federal claims in another 
    forum.
    \item Reversed.
\end{enumerate}

\subsection{Collateral Estoppel: \emph{Levy v. Kosher Overseers Ass'n of Am.}}

To have preclusive effect on issues, the final judgment on the merits must 
have evaluated an identical issue.

\begin{enumerate}
    \item KOA applied to the PTO to register a mark containing a K within a 
    circle. Levy (d.b.a. OK Labs) filed an ``opposition'' with the PTO's 
    Trademark Trial and Appeal Board. The TTAB sustained the opposition and 
    refused KOA's application.
    \item KOA did not appeal the TTAB's decision. It kept using the mark.
    \item OK sued in federal court. The court granted summary judgment against 
    KOA on the collateral estoppel effect of the TTAB's decision and granted a 
    permanent injunction.
    \item The Second Circuit held that there are four factors to apply 
    collateral estoppel to bar litigation:\footnote{Casebook p. 1252.}
    \begin{enumerate}
        \item Identical issues.
        \item Prior proceeding ``must have been actually litigated and 
        actually decided.
        \item Full and fair opportunity for litigation in the prior 
        proceeding.
        \item Final judgment on the merits.
    \end{enumerate}
    \item The court agreed with KOA's argument that the TTAB applied a 
    different test than a federal court would (under the Lanham Act) in 
    evaluating the claims of ``likelihood of confusion.''  The TTAB did not 
    consider actual usage of the mark, whereas a federal court would apply the 
    ``Polaroid factors.''\footnote{Casebook p. 1253.}
    \item Reversed.
\end{enumerate}

\subsection{Informal Proceedings and Issue Preclusion: \emph{Jacobs v. CBS}}

To have an issue-preclusive effect, the earlier proceeding must have had 
adequate procedural safeguards.

\begin{enumerate}
    \item Givens wrote a manuscript for a TV show. He contracted with 
    Westwind to pitch the show to CBS. Under the ``First Agreement,'' CBS 
    agreed to acquire the broadcast rights. Under the ``Second Agreement,'' 
    CBS bought all rights and agreed to credit Webb and Jacobs as Executive 
    Producers.
    \item CBS later produced a show based on a similar premise. It did not 
    credit Givens. The Writers' Guild of America concluded that Givens was not 
    a ``participating writer'' and refused to represent him in arbitration 
    against CBS.
    \item Meanwhile, Givens, Jacobs, Webb, and Westwind sued CBS in California state 
    court. CBS removed to federal court and Givens dropped out.
    \item CBS moved for summary judgment, arguing that the plaintiffs' claims 
    were completely derivative of Givens' claim and that the WGA's 
    determination that Givens was not a participating writer had a collateral 
    estoppel effect. The court granted the motion.
    \item The Ninth Circuit held that to demonstrate collateral estoppel, CBS 
    must show:\footnote{Casebook p. 1258.}
    \begin{enumerate}
        \item Identical issues.
        \item Actual litigation.
        \item Decision on the issue.
        \item Final judgment on the merits.
        \item Plaintiffs wer in privity with Givens.
    \end{enumerate}
    \item The plaintiffs argued that the WGA proceedings could not have an 
    issue-preclusive effect because they lacked adequate procedural 
    safeguards.\footnote{Casebook pp. 1259--1260.} The Ninth Circuit agreed. 
    Reversed.
\end{enumerate}

\subsection{\emph{Taylor v. Sturgell}}

\begin{enumerate}
    \item % TODO
\end{enumerate}

\subsubsection{\emph{Parklane Hosiery v. Shore}}

\begin{enumerate}
    \item % TODO
\end{enumerate}
