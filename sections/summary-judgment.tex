\section{Summary Judgment}

\begin{enumerate}
    \item \textbf{12(b)(6) motion for dismissal for failure to state a claim}: 
    tests the \textbf{sufficiency of the allegations themselves}. Assuming the 
    plaintiff's allegations are true, the court determines whether there is a 
    cause of action. State equivalent is the demurrer. Motions to dismiss are 
    based on the pleadings themselves; any additional factual allegations will 
    cause the motion to be treated as a motion for summary judgment.
    \item \textbf{Rule 56 motion for summary judgment}: tests the 
    \textbf{sufficiency of the \emph{facts supporting} the allegations.} 
    Either side can challenge the legal sufficiency of the other's factual 
    allegations or legal contentions.
    \begin{enumerate}
        \item Granted if the judge determines that ``there is not genuine 
        dispute as to any material fact and the movant is entitled to judgment 
        as a matter of law.'' Rule 56(a).
        \item Exists to decide issues that are so one-sided that a trial would 
        be wasteful.\footnote{Casebook p. 992 n. 1.}
        \item Pleadings do not support motions for summary 
        judgment\footnote{Casebook p. 993 n. 2}---i.e., the motions for 
        summary judgment must be supported with facts, not allegations.
        \item ``...one of the prime uses of discovery is to gather information 
        that will be useful in supporting and opposing summary 
        judgment.''\footnote{Casebook p. 993.}
        \item \textbf{Burden of production}: plaintiff must produce sufficient 
        evidence on each element of the case for a jury to reasonably rule in 
        its favor. Otherwise, under FRCP 50, the judge may grant judgment as a 
        matter of law.\footnote{Casebook p. 994--95 n. 4.}
        \item \textbf{Burden of persuasion}: the standard by which a plaintiff 
        will have to convince a jury (which, in civil cases, is ``by a 
        preponderance'').\footnote{Casebook p. 994 n. 4}
        \item Summary judgment tests the whether a party can meet the burden 
        of production.
        \item \textbf{The moving party must support its motion for summary 
        judgment.} If the moving party could successfully move without showing 
        reason to believe that the other party will not be able to meet the 
        burden of production, the motion for summary judgment would be a 
        discovery device.
    \end{enumerate}
\end{enumerate}

\subsection{FRCP 56}

\begin{itemize}
    \item (a) Court can grant summary judgment if ``there is not genuine 
    dispute as to any material fact and the movant is entitled to judgment as 
    a matter of law.''
    \item (b) Parties may file motions within 30 days after the close of 
    discovery.
    \item (c) Procedures.
    \begin{itemize}
        \item (1) Supporting factual assertions.
        \item (2) Facts must be admissible.
        \item (3) Court can consider materials not cited in the motion.
        \item (4) Affidavits/declarations.
    \end{itemize}
    \item (d) When facts are unavailable to the non-movant.
    \item (e) Failing to properly support or address a fact.
    \item (f) Judgment independent of the motion. [Courts can grant summary 
    judgments absent motions from any party.]
    \item (g) Failing to grant the requested relief. [If the summary judgment 
    doesn't end the case, its outcome can still come into play during trial.]
    \item (h) Court can impose sanctions for bad faith 
    affidavits/declarations.
\end{itemize}

\subsection{Summary Judgment: \emph{Adickes v. S.H. Kress \& Co.}}

To win a motion for summary judgment, \textbf{the moving party must show an 
absence of any genuine issue of fact} when the evidence is viewed in the light 
most favorable to the opposing party.

\begin{enumerate}
    \item Sandra Adickes, a white teacher, took a group of black students to 
    Kress restaurant in Hattiesburg, Mississippi. She was refused service and 
    then arrested for vagrancy.
    \item She alleged that (1) she was refused service because she was part of 
    a mixed-race group and (2) the refusal of service and subsequent arrest 
    resulted from a conspiracy between Kress and the Hattiesburg police. The 
    District Court for the Southern District of New York directed a verdict 
    for the defendants on the first count and granted summary judgment on the 
    second. The Second Circuit unanimously affirmed.
    \item The Supreme Court reversed on both counts (but the edited opinion in 
    the casebook addresses only the summary judgment on the second count).
    \item Justice Harlan:
    \begin{enumerate}
        \item To show conspiracy as alleged, Adickes must show (1) that an 
        employee of Kress deprived her of her constitutional rights and (2) 
        that the defendant acted ``under color of law'' (which is satisfied if 
        an employee and the policeman ``somehow reached an understanding to 
        deny Miss Adickes service''\footnote{Casebook p. 986.}).
        \item Summary judgment was inappropriate because the respondent, 
        Kress, ``failed to carry its burden of showing the absence of any 
        genuine issue of fact''\footnote{Casebook p. 987.} (and any material 
        it submitted ``must be viewed in the light most favorable to the 
        opposing party''\footnote{Casebook p. 988.}).
        \item In this case, the two big factual gaps were that the police 
        officers failed to ``foreclose the possibility'' that they (1) were in 
        the store and (2) influenced the Kress employee to not serve 
        Adickes.\footnote{Casebook p. 990.}
        \item Because respondent did not meet the burden of establishing the 
        police officers' presence, petitioner was not required to file 
        opposing affidavits.
        \item Reversed.
    \end{enumerate}
\end{enumerate}

\subsection{\emph{Celotex Corp. v. Catrett}}

The movant for summary judgment bears the initial burden of production, i.e., 
he must clearly show that there is no factual dispute. If the responding party 
will bear the burden of persuasion at trial, the moving party need only show 
that the record contains no evidence that the nonmovant will be able to prove 
an essential element of its case.

More simply: the defendant can successfully claim that plaintiff's case is 
unfounded if there is no evidence in the record at that time that the 
plaintiff can make its case.

\begin{enumerate}
    \item Catrett died in 1979. In 1980, his wife (respondent) filed suit 
    against 15 asbestos companies, including Celotex (petitioner).
    \item Celotex moved for summary judgment for lack of evidence showing that 
    its product was a proximate cause of Catrett's death, including a lack of 
    witnesses who could testify to that effect. Catrett produced three 
    documents she claimed demonstrated ``a genuine material factual 
    dispute.''\footnote{Casebook p. 996--97.} Celotex argued that the 
    documents were inadmissible hearsay.
    \item The District Court for D.C. granted the motions from Celotex and the 
    other defendants. Catrett appealed only the grant for Celotex. The D.C. 
    Circuit reversed on the grounds that Celotex failed to show any evidence 
    to support its motion.\footnote{Casebook p. 997.}
    \item Justice Rehnquist:
    \begin{enumerate}
        \item The D.C. Circuit's opinion was inconsistent with FRCP 56. The 
        plaintiff bears the burden of proof for its claim, and in this case 
        the defendant's motion contended that the plaintiff failed to 
        establish the existence of an element essential to its case.
        \item The moving party need not negate its opponent's claim.
        \item ``One of the principle purposes of the summary judgment rule is 
        to isolate and dispose of factually unsupported claims or defenses, 
        and we think it should be interpreted in a way that allows it to 
        accomplish this purpose.''\footnote{Casebook p. 998.}
        \item Reversed.
    \end{enumerate}
\end{enumerate}

\subsection{\emph{Arnstein v. Porter}}

[Copyright infringement suit against Cole Porter. We skipped it in class.]

\subsection{Disagreement about the Existence of a Dispute of Material Fact: 
\emph{Scott v. Harris}}

Courts sometimes disagree about whether there is a genuine dispute of material 
fact. Courts have broad discretion to evaluate whether a dispute exists.

\begin{enumerate}
    \item Harris fled from the police in a high-speed car chase. Deputy Scott 
    bumped Harris's car during the chase, causing him to crash and become 
    quadriplegic.
    \item Harris sued in district court for violation of the Fourth Amendment. 
    The legal question was whether a police officer can ``take actions that 
    place a fleeing motorist at risk of serious injury or death in order to 
    stop the motorist's flight from endangering the lives of innocent 
    bystanders.''\footnote{Casebook p. 1015.}
    \item Scott filed a motion for summary judgment based on qualified 
    immunity. The district court denied, holding that there were significant 
    disagreements about issues of fact which required submission to a jury.
    \item The Eleventh Circuit affirmed.
    \item Justice Scalia:
    \begin{enumerate}
        \item The standard for reviewing motions for summary judgment is to 
        view the evidence in the light most favorable to the nonmoving party.
        \item Here, however, a videotape ``quite clearly contradicts the 
        version of the story told by respondents.''\footnote{Casebook p. 1017.}
        \item Eight Supreme Court justices believed the video obviously showed 
        that Harris posed an imminent threat. Therefore, Scott obviously had a 
        qualified immunity to use deadly force and his summary judgment motion 
        should have been granted.
        \item Reversed.
    \end{enumerate}
    \item Justice Stevens: the video shows that Harris was \emph{not} 
    endangering others. In any case, the district court and the appellate 
    court are more familiar with the local roads.
\end{enumerate}
