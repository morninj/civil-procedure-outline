\section{Appeals}

\begin{enumerate}
    \item The right to appellate review exists when (1) the court
    violated procedure or (2) the court's decision rested on ``misapplication 
    of the substantive law or gross misapprehension of the 
    facts.''\footnote{Casebook p. 1325.}
    \item If a plaintiff received less than he wanted at trial, he becomes an 
    ``aggrieved party'' and can appeal the judgment.
    \item State courts often hear mooted cases to decide important questions 
    of law. Federal courts observe stricter standards on mooted cases because 
    of Article III's cases and controversies restriction.
    \item Parties that have intervened in litigation have a right to appeal.
    \item \textbf{Final judgment rule}: appeal is generally allowed only from 
    final judgment, though interlocutory appeal is allowed under some 
    circumstances. For instance, FRCP 23(f) allows interlocutory appeal of 
    class certification decisions.\footnote{Casebook p. 1326.}
    \item Scope of appeal:\footnote{Casebook p. 1326--1327.}
    \begin{enumerate}
        \item Questions on appeal generally must have been raised during 
        trial.
        \item Appellant must specify matters on appeal.
        \item Appellate courts do not accept new evidence.
        \item Appellate courts will not overrule decisions within the ``sound 
        discretion'' of the trial judge unless abuse of discretion is 
        apparent.
        \item Harmless or nonprejudicial errors will not be reversed.
        \item The scope of injunctive relief is generally limited to abuse of 
        discretion.
    \end{enumerate}
\end{enumerate}

\subsection{\emph{Digital Equip. Corp. v. Desktop Direct}}

\begin{enumerate}
    \item Desktop sued Digital for use of the ``Desktop Direct'' name. The two 
    reached a confidential settlement which included a waiver of all damages 
    and dismissal of the suit. Desktop filed a notice of dismissal in district 
    court.
    \item Months later, Desktop discovered that Digital had misrepresented 
    material facts during settlement negotiations. It moved to vacate the 
    dismissal. The court vacated the dismissal and Digital appealed.
    \item The Tenth Circuit dismissed the appeal for lack of jurisdiction 
    under 28 U.S.C. \S\ 1291, holding that the district court's order to 
    vacate the dismissal neither ended the litigation nor fell within the 
    ``collateral order exception.
    \item Justice Souter:
    \begin{enumerate}
        \item The \textbf{collateral order doctrine} allows appeal from ``a 
        narrow class of decisions that do not terminate the litigation, but 
        must...nonetheless be treated as `final.'''\footnote{Casebook p. 
        1347.}
        \item Digital argued that its ``right not to stand trial'' under its 
        private settlement agreement required the protection of immediate 
        appeal. It argued that its right was analogous to the qualified 
        immunity right in \emph{Mitchell}.
        \item The Supreme Court held that the right Digital asserted can be 
        adequately protected once final judgment had been rendered at trial. 
        If it allowed Digital's interlocutory appeal, other types of decision
        would also be subject to interlocutory appeal---including claims of 
        lack of personal jurisdiction, that the statute of limitations has 
        run, that no material fact is in dispute, and 
        others.\footnote{Casebook p. 1349.}
        \item ``...such a right by agreement does not rise to the level of 
        importance needed for recognition under \S\ 1291.''\footnote{Casebook 
        p. 1350.}
        \item Affirmed.
    \end{enumerate}
\end{enumerate}

\subsection{28 U.S.C. \S\ 1257}

\begin{itemize}
    \item (a) The Supreme Court can review decisions from state courts of last 
    resort.
\end{itemize}

\subsection{28 U.S.C. \S\ 1291}

Federal appellate courts can hear appeals ``from all final decisions of the 
district courts of the United States...''

\subsection{28 U.S.C. \S\ 1292}

\begin{itemize}
    \item (a)(1) Federal appellate can hear appeals from interlocutory orders 
    from district courts.
    \item (b) Appellate courts can decide whether an order from a district 
    court ``involves a controlling question of law as to which there is a 
    substantial ground for for difference of opinion and that an immediate 
    appeal from the order may materially advance the ultimate termination of 
    the litigation...''
    \item [Do district court judges have to approve parties' requests for 
    interlocutory appeal?]
\end{itemize}
