\section{Class Actions}

\begin{enumerate}
    \item Class actions can involve either plaintiff classes (common) or
    defendant classes (rare).
    \item Binding determinations can be made upon class members who are
    absent, unnamed, and sometimes unnotified.
    \item ``...in almost all class actions the attorney's financial
    investment, ideological stake in the outcome, and potential to influence
    the conduct of the case is much greater than that of the named class
    representative.''\footnote{Casebook p. 799 n. 1.d.}
\end{enumerate}

\subsection{FRCP 23: Class Actions}

\begin{enumerate}
    \item Rule 23 was revised in 1966 with three goals in
    mind:\footnote{Casebook p. 797.}
    \begin{enumerate}
        \item Define cases where the benefits of a class suit outweigh the
        disadvantages.
        \item Specify that all class suits are binding and define the scope of
        their preclusive effect.
        \item Ensure maximum advantage and fair representation for absent
        class members.
    \end{enumerate}
    \item The rule's specific provisions are:
    \begin{itemize}
        \item (a) Requirements applicable to all class actions.
        \begin{itemize}
            \item (1) Numerosity: the class must be so numerous that joinder
            is impracticable.
            \item (2) Commonality: there must be questions of law or fact
            common to the class (a ``not particularly stringent
            requirement''\footnote{Casebook p. 798 n. 1.b.}).
            \item (3) Typicality: the claims or defenses of the representative
            party must be typical of those of the class as a whole.
            \item (4) Fair and adequate protection of the interests of the
            class: the named parties must represent the entire class's
            interest---e.g., it must avoid conflicts of interest, and the
            class must not include groups with ``sharply differing
            interests.''footnote{Casebook p. 799.} The representation by the
            class attorney must also be adequate.
        \end{itemize}
        \item (b) Types of class actions.
        \begin{itemize}
            \item (1) Class treatment is allowed when (A) individual suits would
            result in incompatible standards of conduct for the non-class or
            (B) individual suits would impair the ability of those who have
            not brought individual suits, e.g., in ``limited fund'' suits
            where the fund is insufficient to adequately cover the number of
            possible individual claims. Generally limited to suits seeking
            injunctive or declaratory relief.
            % todo: examples of 23(b)(1)(A) and (B) suits?
            \item (2) Class treatment is allowed when the party opposing the
            class has ``acted or refused to act on grounds that apply
            generally to the class.'' Civil rights suits are the most common.
            \item (3) Class treatment is allowed when questions common to the
            class ``predominate'' over questions affecting individual class
            members, and class action must be ``superior'' to other methods of
            adjudication. Notice to class members is mandatory and members
            must have the option of opting out of the class (unlike (b)(1) and
            (2) actions).
        \end{itemize}
    \end{itemize}
\end{enumerate}

\subsection{\emph{Chandler v. Southwest Jeep Eagle}}

\begin{enumerate}
    \item % todo
\end{enumerate}

