\section{Jurisdiction}

TODO: difference between territorial jurisdiction and personal jurisdiction?

\subsection{\emph{Pennoyer}: Personal Jurisdiction}

\begin{enumerate}
    \item  In an in personam case, does service by publication to a non-resident defendant establish territorial jurisdiction? \emph{Pennoyer v. Neff}:
Mitchell sues Neff in Oregon state trial court for nonpayment of legal fees rendered in 1862-1863. Neff is nowhere to be found, so Mitchell publishes notice of the suit in a newspaper. Neff does not appear, so the court orders a default judgment. The property is attached and then sold to Mitchell, who sells it to Pennoyer. Eight years later, Neff successfully sues Pennoyer to recover the property. The court (Justice Field) relies on an analytical framework in which the basis for jurisdiction is a state's territorial power. States are all-powerful within their borders, and powerless beyond. It holds that service by publication isn't good enough for in personam suits against a non-resident (though it might be good enough for in rem suits). Thus, the original judgment against Neff was void.
    \item Bradt: a judgment void when rendered is void forever.
\end{enumerate}

\subsection{\emph{Hess v. Pawloski}}

\begin{enumerate}
    \item Can a state implement a statute that requires out-of-state drivers to give implied consent to jurisdiction within that state?
    \item Plaintiff, a Pennsylvania resident, ``negligently and wantonly drove a motor vehicle on a public highway in Massachusetts,'' causing injury to the defendant. In a MA Superior Court, plaintiff contested MA's jurisdiction, which was denied. The Supreme Judicial Court upheld the order. Plaintiff appealed to the Supreme Court on Fourteenth Amendment grounds. The court reasoned that earlier cases (e.g., \emph{Kane v. New Jersey} have upheld the constitutionality of statutes that require out-of-state drivers to appoint an agent to receive process before using the highway. States can legitimately require to appoint similar agent implicitly, and these kinds of statutes do not not constitute discrimination against non-residents. Therefore, it is consistent with the Due Process Clause for states to require out-of-state drivers to implicitly appoint an agent to receive process, thereby establishing jurisdiction over those drivers if civil actions arise.



\end{enumerate}

\subsection{\emph{International Shoe}}

\begin{enumerate}
    \item TODO
\end{enumerate}

\subsection{\emph{World-Wide Volkswagen Corp. v. Woodson}}

\begin{enumerate}
    \item TODO
\end{enumerate}

\subsection{\emph{Burger King Corp. v. Rudzewicz}}

\begin{enumerate}
    \item TODO
\end{enumerate}

\subsection{\emph{Goodyear Dunlop Tires Operations, S.A. v. Brown}}

\begin{enumerate}
    \item TODO
\end{enumerate}

\subsection{\emph{J. McIntyre Machinery, Ltd. v. Nicastro}}

\begin{enumerate}
    \item TODO
\end{enumerate}

\subsection{General and Specific Jurisdiction}

\begin{enumerate}
    \item TODO (pp. 185-189)
\end{enumerate}

\subsection{Purposeful Availment and Purposeful Direction}

\begin{enumerate}
    \item TODO (pp. 237-241)
    \item \textbf{\emph{Calder v. Jones}}: TODO pp. 238--239
\end{enumerate}

\subsection{\emph{Benusan Restaurant Corp v. King}}

\begin{enumerate}
    \item TODO
\end{enumerate}

TODO: pp. 259-261 notes 2-5; Rulebook pp. 24-25: FRCP 4(k)

\subsection{\emph{Shaffer v. Heitner}}

\begin{enumerate}
    \item TODO
\end{enumerate}

\subsection{\emph{Burnham v. Superior Court of Calif.}}

\begin{enumerate}
    \item TODO
\end{enumerate}

