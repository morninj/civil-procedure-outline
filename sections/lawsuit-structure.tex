\section{Structure of a Lawsuit}

\subsection{Preliminaries}

\begin{enumerate}
    \item Find a lawyer.
    \item Learn the facts.
    \item Determine the dispute and remedy.
\end{enumerate}

\subsection{Which Court?}

\begin{enumerate}
    \item \emph{Territorial jurisdiction}: there must be a minimal level of contact between the defendant and the court's territorial sovereign (e.g., the state).
    \item \emph{Subject matter jurisdiction}: federal courts have higher thresholds (e.g., interstate disputes in amounts above \$75,000). In cases of overlap, plaintiff can choose.
    \item \emph{Venue}: usually must have some connection to the place where the dispute occurred.
\end{enumerate}

\subsection{Complaint, Filing, and Service of Process}

\begin{enumerate}
    \item \emph{Complaint}: plaintiff's statement of claims. Sometimes called petition or declaration.
    \item \emph{Filing}: file complaint at the courthouse. This is when the suit commences.
    \item \emph{Summons}: served to each defendant.
\end{enumerate}

\subsection{Responding to the Complaint}

\begin{enumerate}
    \item Preliminary objections
    \begin{enumerate}
        \item E.g., disputes over territory or venue.
        \item Motions (to dismiss or quash).
        \item \emph{Memorandum of law}: the legal arguments supporting a request.
    \end{enumerate}
    \item \emph{Default judgment}: occurs if the defendant does nothing. Can be set aside if justified.
    \item Pleading in response to the complaint.
    \begin{enumerate}
        \item In complaints, defendant don't try to prove their case--only to assert what he hopes can be proved.
        \item \emph{Cause of action}: the violation of law in question.
        \item \emph{General demurrer}: ``even if you're right, you're not entitled to recover anything.'' I.e., so what?
        \item By default, defendants are deemed to admit allegations they don't deny.
    \end{enumerate}
    \item Defendant's Answer:
    \begin{enumerate}
        \item \emph{Negative defenses}: contesting the facts.
        \item \emph{Affirmative defenses}: contending other factual circumstances.
        \item \emph{Motion to strike}: e.g., if plaintiff thinks defendant's answer is insufficient in point of substantive law.
    \end{enumerate}
    \item Some jurisidictions allow the plaintiff to make a reply to the defendant's answer; otherwise, the answer is deemed denied by default.
\end{enumerate}

\subsection{Discovery, Summary Judgment, Settlement}

\begin{enumerate}
    \item \emph{Discovery}: each side investigates its opponent's claims.
    \item \emph{Interrogatories}: written questions.
    \item \emph{Request for production}: documents, opportunities to inspect, other relevant items.
    \item \emph{Depositions}: Party questions a witness on camera and/or before a court reporter.
    \item \emph{Summary judgment}: Can be granted if something crucial can be determined beyond legitimate dispute.
    \item \emph{Affidavit}: Sworn statement.
    \item \emph{Pretrial conference}: attempt to resolve the dispute before litigating.
\end{enumerate}

\subsection{Trial}

\begin{enumerate}
    \item \emph{At law}: dor damages; can be tried by a jury.
    \item \emph{In equity}: e.g., for an injunction; normally triable without jury.
    \item Jury selection: voire dire, challenged for cause, limited number of peremptory challenges
    \item Trial process:
    \begin{enumerate}
        \item Opening statements
        \item Case in chief (plaintiff)
        \item Direct and cross examination of witnesses
        \item Plaintiff rests
        \item Defendants can request judgment as a matter of law if they believe the claim is invalid
        \item Case in chief (defendant)
        \item Adverse witness: plaintiff himself is called
        \item Either side can again call for a judgment as a matter of law
        \item Closing argument (plaintiff)
        \item Closing argument (defendant)
        \item Judge instructs jury; jury deliberates and returns verdict
        \item Jury often (but not always) must be unanimous
    \end{enumerate}
\end{enumerate}

\subsection{Post-Trial or Post-Judgment Motions}

\begin{enumerate}
    \item \emph{Non obstante veredicto}: judgment notwithstanding the verdict, e.g., in response to earlier motions for judgment as a matter of law.
    \item Parties can seek a new trial on the basis of procedural errors
\end{enumerate}

\subsection{Appeal}

\begin{enumerate}
    \item Can only happen after final judgment, even if there's a gross error early in the process
    \item \emph{Interlocutory appeal}: in some jurisdictions, appeal can be made before final judgment
    \item \emph{Mandamus}: requires the judge to do something
    \item \emph{Prohibition}: on the judge; usually comes from an appellate court in the form of a writ of prohibition
    \item \emph{Reversible error}: something on which an appellate court can reverse a decision and call a new trial
    \item \emph{Harmless error}: didn't affect the outcome of the trial
    \item Appellate review is almost always on the basis of law, not on fact, unless there is ``no substantial evidence'' to support a factual determination
    \item Appellate court will usually only consider objections that were raised in the trial court
\end{enumerate}
