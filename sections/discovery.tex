\section{Discovery}

\begin{enumerate}
    \item Discovery consists of (1) interrogatories, (2) requests for production or inspection, and (3) depositions.\footnote{Casebook p. 9.}
    \item Fuller disclosure leads to the most favorable case for each party.\footnote{Casebook p. 882.}
    \item Legitimate purposes: promotes settlement, helps determine whether a case can be decided in a summary judgment.
    \item Less legitimate purposes: inflict costs, harass, reconstruction [--? see p. 883.]
    \item State discovery rules generally track the federal rules.
    \item The discovery process:
    \begin{enumerate}
        \item \emph{Informal investigation}: happens outside the compulsory structure of formal discovery---interviews, document review, property visits.
        \item \emph{Discovery plan}: FRCP 26(f) requires parties to agree to a discovery plan.
        \item \emph{Initial disclosures}: mandatory disclosures include (1) names and contact details of relevant individuals, (2) copies or descriptions of records, (3) computations of damages, and (4) insurance information. FRCP 26(a)(1)(A). Parties are only required to disclose information that is favorable to their cases.\footnote{Casebook p. 885.}
        \item \emph{Depositions}:
        \begin{enumerate}
            \item Depositions are binding but expensive. Lawyers generally depose all unfriendly witnesses.\footnote{Casebook p. 886.}
            \item FRCP 30 defines the scope of depositions.
            \item Lawyers can instruct defendants not to answer to (1) preserve a privilege, (2) enforce a protect order limiting discovery, or (3) stop abusive behavior.
        \end{enumerate}
        \item \emph{Interrogatories}: written questions that must be answered under oath, often with accompanying requests for documents. FRCP 33 governs interrogatories.
        \item \emph{Production}: items can be obtained by subpoena if necessary. FRCP 34 controls.
        \item \emph{Physical and mental examinations}: only when physical or mental states are issues in the case. FRCP 35 controls.
        \item \emph{Requests for admission}: to determine whether facts are accurate and documents genuine. FRCP 36.
        \item \emph{Motions for protective orders and motions to compel}: court must award attorneys' fees to the winning party. FRCP 26(c), 37(a)(1).
        \item \emph{Sanctions}: most commonly an award of costs.
    \end{enumerate}
\end{enumerate}

\subsection{FRCP 26: Duty to Disclose; General Provisions Governing Discovery}

\begin{itemize}
    \item (a) Required disclosures.
    \begin{itemize}
        \item (1) Initial disclosure.
        \begin{itemize}
            \item (A) Generally:
            \begin{itemize}
                \item (i) Names and contact details.
                \item (ii) Copies or descriptions of documents, etc.
                \item (iii) Computation of damages.
                \item (iv) Insurance details.
            \end{itemize}
            \item (E) Parties must supplement disclosures when required under 26(e).
        \end{itemize}
    \end{itemize}
    \item (b) Discovery scope and limits.
    \begin{itemize}
        \item (1) Parties can discover all nonprivileged matter relevant to claims or defenses. It need not be admissible at trial if it might lead to the discovery of admissible evidence.
        \item (2) Limitations on frequency and extent.
        \begin{itemize}
            \item (A) Court can alter limits.
            \item (B) Electronic information need not be produced if it carries undue burden or cost.
            \item (C) Court must limit discovery under certain circumstances (enumerated within).
        \end{itemize}
    \end{itemize}
    \item (c) With good cause, the court can protect material from discovery.
    \item (d) After the initial discovery meeting under 26(f), discovery can proceed in any sequence.
    \item (e) Supplements and corrections are sometimes required.
    \item (f) Conference of the parties; planning for discovery.
    \begin{itemize}
        \item (1) Parties must confer as soon as practicable.
        \item (2) Parties must submit a discovery plan within 14 days.
        \item (3) Requirements for the discovery plan.
    \end{itemize}
\end{itemize}

\subsection{FRCP 29: Stipulations about Discovery Procedure}

\subsection{FRCP 30: Depositions by Oral Examination}

\begin{itemize}
    \item (a) When a deposition may be taken.
    \item (b) Notice of the deposition; other formal requirements.
    \begin{itemize}
        \item (1) Notice.
        \item (2) Producing documents.
        \item (6) Notice or subpoena directed to an organization.
    \end{itemize}
    \item (c) Examination and cross-examination; record of the examination; written questions.
    \begin{itemize}
        \item (1) Conducted as they would be at trial.
    \end{itemize}
    \item (d) Duration; sanction; motion to terminate or limit.
    \begin{itemize}
        \item Limited to one day of seven hours.
    \end{itemize}
\end{itemize}

\subsection{FRCP 33--36}

\begin{itemize}
    \item 33: Interrogatories to Parties.
    \item 34: Producing Documents, Electronically Stored Information, and Tangible Things, or Entering onto Land, for Inspection and Other Purposes.
    \item 35: Physical and mental examinations.
    \item 36: Requests for admission.
\end{itemize}

\subsection{Privileges and Sanctions}

\subsubsection{\emph{Hinckman v. Taylor}}

\begin{enumerate}
    \item % TODO
\end{enumerate}

\subsubsection{Work-product doctrine}

\begin{enumerate}
    \item % TODO 928-31 nn 1-4
    \item % TODO 975--978: nn 5-6
    \item % TODO see FRCP 26(b)(3)
\end{enumerate}


\subsubsection{FRCP 37}

\begin{itemize}
    \item (a) % TODO
    \begin{itemize}
        \item (1) % TODO
    \end{itemize}
    \item (b) % TODO
\end{itemize}
