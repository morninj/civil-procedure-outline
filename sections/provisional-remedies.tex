\section{Provisional Remedies}

\subsection{Statute of Limitations: \emph{United States v. Kubrick}}

 The statute of limitations tolls when the plaintiff becomes aware of the 
 existence and cause of his injury, not when he becomes aware of malpractice.

\begin{enumerate}
    \item Kubrick was rendered partially deaf from neomycin treatment at a
    Veterans Administration hospital. He discovered the possibility of 
    malpractice only after the two year statute of limitations had expired.
    \item The precise issue was whether the claim accrues when the plaintiff 
    is aware of the existence and cause of his injury or when he is also aware 
    of the possibility of malpractice.
    \item Justice White: there's a clear rule here. The statute of limitations 
    tolls when the plaintiff becomes aware of the existence and cause of his 
    injury.
    \item Justice Stevens, dissenting: a rigid rule is unnecessary---all we 
    need is a looser standard that can be applied on a case-by-case basis.
    \item Why does the statute of limitations exist?\footnote{Casebook pp.  
    54--55}
    \begin{enumerate}
        \item Protect against the ``cloud of litigation.''
        \item Protect against ``stale claims.''
        \item Keep the plaintiff from sitting on his rights.
    \end{enumerate}
\end{enumerate}

\subsection{Due Process Requirements}

\begin{enumerate}
    \item Fifth Amendment: ``No person shall be\ldots deprived of life, 
    liberty, or property, without due process of law.''
    % TODO: check CMOS ellipsis style and replace throughout
    \item Fourteenth Amendment: ``No \textbf{State} shall ... deprive any 
    person of life, liberty, or property, without due process of law.''
\end{enumerate}

\subsection{Remedies}

\begin{enumerate}
    \item \textbf{Plenary}: Usually awarded at the end of a lawsuit. Usual 
    types: compensatory and punitive damages, injunctions, and declaratory 
    judgments.
    \item \textbf{Provisional}: Can be awarded at any time while a lawsuit is 
    pending. Usual types: attachment (seizure of property), temporary 
    restraining orders, preliminary injunctions. They are ``designed to 
    stabilize the situation pending the final disposition of the case or to 
    provide security to the plaintiff so that if she succeeds in obtaining 
    judgment she will be able to enforce it effectively.''\footnote{Casebook 
    p. 46.}
\end{enumerate}

\subsection{Notice and Opportunity to be Heard: \emph{Fuentes v. Shevin}}

Ex parte prejudgment and seizure requires notice and opportunity to be heard.

\begin{enumerate}
    \item Do statutes that allow writs of replevin only upon ex parte 
    application and posting of bond violate the Fourteenth Amendment?
    \item In multiple consolidated cases, a creditor was granted an ex parte 
    writ of replevin for the property of Fuentes, debtor in default (which 
    statutes in Florida and Pennsylvania statutes allowed).
    \item Justice Stewart: these statutes allowing ex parte attachment violate 
    the Due Process Clause (Fourteenth Amendment). Absent extraordinary 
    circumstances, due process requires \textbf{notice} and 
    \textbf{opportunity to be heard} before deprivation.
    \item Justice White, dissenting: ``If there is a default, it would seem 
    not only `fair,' but essential, that the creditor be allowed to repossess; 
    and I cannot say that the likelihood of a mistaken claim of default is 
    sufficiently real or recurring to justify a broad constitutional 
    requirement that a creditor do more than the typical state law requires 
    him to do.''\footnote{Casebook p. 74.}
    \item Key due process protections: \textbf{notice} and \textbf{opportunity 
    to be heard} (in a meaningful way).
    \item \emph{Mitchell v. W.T. Grant}: A similar statute in Louisiana was 
    upheld on the grounds that (1) the applicant for the writ must declare the 
    facts in a certified petition or affidavit, and (2) the showing must be 
    made to a judge, not merely a court official.
    \item \emph{North Georgia Finishing, Inc. v. Di-Chem, Inc.}: A similar 
    Georgia statute was struck down because (1) the affidavit can be filed by 
    the petitioner's attorney, who need not have any direct knowledge of the 
    facts of the dispute, and (2) the writ is issuable by a court clerk, not a 
    judge.
    \item The minimum constitutional requirements for valid ex parte 
    prejudgment and seizure appear to be:
    \begin{enumerate}
        \item An application grounded in facts.
        \item Issued by a judge, not a court official.
        \item A speedy hearing.
        \item Only applicable to a limited range of transactions.\footnote{See 
        California's version of these statues, casebook p. 82 top.}
    \end{enumerate}
\end{enumerate}

\subsection{Deprivation and Due Process: \emph{Connecticut v. Doehr}}

Does prejudgment attachment of real estate without prior notice 
or hearing, without extraordinary circumstances, and without a 
bond violates due process.
% TODO: replace glyphs ’ with non-glyph apostrophe ' throughout

\begin{enumerate}
    \item  Under a Connecticut statute, DiGiovanni won a \$75,000 prejudgment 
    attachment on Doehr’s home in conjunction with a civil action for assault 
    and battery.
    \item Justice White: the Connecticut statute would allow deprivation for 
    cases where the defendant's property claim would fail to convince a jury. 
    Without exigent circumstances, a preattachment hearing is required.
    \item Justices Marshall, Stevens, O'Connor, and White, concurring: bonds 
    are also necessary in all cases.
    \item Justice Rehnquist, concurring: liens can serve a useful purpose 
    (e.g., for laborers to enforce their interests over delinquent 
    landowners). Also, the terms ``bond'' and ``exigent circumstances'' are 
    overly vague.
    \item The \emph{Matthews} test determines whether deprivation meets due 
    process requirements:\footnote{Casebook p. 85.}
    \begin{enumerate}
        \item What are the private interests that deprivation will affect?
        \item What is the risk of erroneous deprivation, and what safeguards 
        are in place?
        \item What is the interest of the party seeking the judgment remedy 
        (in \emph{Mathews} originally, this was the government; here it's the 
        private plaintiff)?
    \end{enumerate}
\end{enumerate}
